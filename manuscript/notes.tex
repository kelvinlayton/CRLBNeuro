\documentclass{article}


\usepackage{amsmath}
\usepackage{amssymb}
\usepackage{graphicx}
\usepackage{color}
\usepackage{subfig}
\usepackage{float} % For [H] placement

\usepackage[ruled,vlined,linesnumbered]{algorithm2e}

\usepackage{booktabs}

\usepackage[bookmarks=true,ocgcolorlinks=true,plainpages=false, breaklinks=true, bookmarksopen=true, bookmarksnumbered=true]{hyperref}
%\hypersetup{bookmarks=false}  %hide the bookmarks bar
%\hypersetup{bookmarksopen=false}  % expand tree of bookmarks or just show first level
\hypersetup{linkcolor=blue, citecolor=magenta,urlcolor=blue} % electronic
\hypersetup{colorlinks=true}

\usepackage[T1]{fontenc}
\usepackage{palatino}

\allowdisplaybreaks

\renewcommand{\vec}[1]{\ensuremath{{\boldsymbol #1}}}
\newcommand{\mat}[1]{\ensuremath{\boldsymbol{#1}}}

\newcommand{\unit}[1]{\textcolor{green}{#1}}
\newcommand{\brian}[1]{\textcolor{blue}{#1}}

\title{Notes on a Sigmoidal Transformation of a Gaussian}
\date{\today}

\begin{document}
\maketitle



\subsection*{Definition of Sigmoid}
The error function version of the sigmoid is 
\begin{align}
	g(x) &= \frac{1}{2}\left(\mathrm{erf}\left(\frac{x-v_{0}}{\sqrt{2}\varsigma}\right) + 1\right).
\end{align}
This sigmoid is the cdf of a Gaussian, where $v_0$ is the mean and $\varsigma$ is the variance.

\subsection*{Transformation of GRV By erf}
Let
\begin{align}
	x\sim\mathcal{N}\left(\mu,\sigma^2\right)
\end{align}
and
\begin{align}
	y =& g(x) \\
	=& \frac{1}{2}\left(\mathrm{erf}\left(\frac{x-v_0}{\sqrt{2}\varsigma}\right) + 1\right).
\end{align}
Define some quantities to get transformed distribution:
\begin{align}
	x =& g^{-1}(y) \\
	  =& v_0 + \sqrt{2}\varsigma\mathrm{erf}^{-1}\left(2y-1\right) \\
	\frac{\mathrm{d}x}{\mathrm{d}y} =& \sqrt{2\pi}\varsigma\exp \left( \left[\mathrm{erf}^{-1}\left(2y-1\right)\right]^2\right)
\end{align}
The pdf of $y$ is
\begin{align}
	f_Y(y) =& \left|\frac{\mathrm{d}x}{\mathrm{d}y}\right|f_X(g^{-1}(y)) \\
	=& \frac{\varsigma}{\sigma}\exp\left(\left[\mathrm{erf}^{-1}\left(2y-1\right)\right]^2\right)\exp\left(-\frac{\left(v_0 + \sqrt{2}\varsigma\mathrm{erf}^{-1}\left(2y-1\right) - \mu\right)^2}{2\sigma^2}\right)
\end{align}
for $0<y<1$. Now let
\begin{align}
	z = \mathrm{erf}^{-1}\left(2y-1\right)
\end{align}
giving
\begin{align}
	f_Y(y)=& \frac{\varsigma}{\sigma}\exp\left(z^2\right)\exp\left(-\frac{\left(v_0 + \sqrt{2}\varsigma z - \mu\right)^2}{2\sigma^2}\right) \\
	=& \frac{\varsigma}{\sigma}\exp\left(-\frac{2\sigma^2 z^2 - \left(v_0 + \sqrt{2}\varsigma z - \mu\right)^2}{2\sigma^2}\right)
\end{align}
Get the numerator of the exponent in quadratic form
\begin{align}
	\mathrm{num} =& 2\left(\varsigma^2 - \sigma^2\right)z^2 + 2\sqrt{2}\varsigma\left(v_0-\mu\right)z + \mu^2 + v_0^2 - 2\mu v_0 \\
	=& az^2 + bz + c \\
	=& a\left(z-h\right)^2 + k
\end{align}
where
\begin{align}
	a =& 2\left(\varsigma^2 - \sigma^2\right) \nonumber \\
	b =& 2\sqrt{2}\varsigma\left(v_0-\mu\right) \nonumber \\
	c =& \mu^2 + v_0^2 - 2\mu v_0 \nonumber \\
	k =& c - \frac{b^2}{4a}\nonumber \\
	h =& - \frac{b}{2a}\nonumber
\end{align}
This gives
\begin{align}
	f_Y(y)=& \frac{\varsigma}{\sigma}\exp\left(\frac{k}{2\sigma^2}\right)\exp\left(-\frac{ (z - h)^2}{\frac{2\sigma^2}{a}}\right)
\end{align}
So we have a Gaussian like structure that may be exploited if we linearize $z$ (via first-order Maclaurin series expansion) such that
\begin{align}
	z =& \mathrm{erf}^{-1}\left(2y-1\right) \\
	\approx& \frac{\sqrt{\pi}}{2}\left(2y-1\right)
\end{align}
This may give and interesting alternative for Gaussian approximation filter??

\subsection*{Mode Derivation}
The Gaussian transformed by the error function sigmoid has the pdf
\begin{align}
	f_X(x)=& a\exp\left(-\frac{ \left(\mathrm{erf}^{-1}\left(2x-1\right) - b\right)^2}{c}\right)
\end{align}
Now let's find the mode. To simplify, first we take $\ln$ of $f_X(x)$ (since this will not shift the max) giving
\begin{align}
	\ln f_X(x)=& \ln a - \frac{ (\mathrm{erf}^{-1}\left(2x-1\right) - b)^2}{c}
\end{align}
Now find max by taking derivative and setting to zero. To do this we define
\begin{align}
	v =& 2x-1 \nonumber \\
	w =& \mathrm{erf}^{-1}(v) \nonumber \\
	y =& w-b \nonumber \\
	z =& y^2/c \nonumber
\end{align}
\begin{align}
	\frac{\mathrm{d}}{\mathrm{d}x} \ln f_X(x) =& 0
\end{align}
where
\begin{align}
	\frac{\mathrm{d}z}{\mathrm{d}y} \frac{\mathrm{d}y}{\mathrm{d}w} \frac{\mathrm{d}w}{\mathrm{d}v} \frac{\mathrm{d}v}{\mathrm{d}x} =& \frac{2y}{c} \times 1 \times \frac{\sqrt{\pi}}{2}\exp\left[\left(\mathrm{erf}^{-1}(v)\right)^2\right] \times 2 \\
	=& 0.
\end{align}
This is true where $y=0$. This implies the mode occurs at
\begin{align}
	\mathrm{erf}^{-1}(2x-1) = b.
\end{align}
From above
\begin{align}
	b = \frac{\sqrt{2}\varsigma\left(\mu-v_0\right)}{2\left(\varsigma^2-\sigma^2\right)}.
\end{align}
Therefore the mode is
\begin{align}
	\mathrm{mode}\left(f_X(x)\right) = \frac{1}{2}\left(\mathrm{erf}\left(\frac{\sqrt{2}\varsigma\left(\mu-v_0\right)}{2\left(\varsigma^2 - \sigma^2\right)}\right)+1\right)
\end{align}
This is correct and can be checked by code below.

% \subsection{New Gaussian Approximation}
% \begin{align}
% 	\mathrm{num} =& 2\left(\varsigma^2 - \sigma^2\right)z^2 + 2\sqrt{2}\varsigma\left(v_0-\mu\right)z + \mu^2 + v_0^2 - 2\mu v_0 \\
% 	=& 2\left(\varsigma^2 - \sigma^2\right)(\frac{\sqrt{\pi}}{2}\left(2y-1\right))^2 + 2\sqrt{2}\varsigma\left(v_0-\mu\right)\frac{\sqrt{\pi}}{2}\left(2y-1\right) + \mu^2 + v_0^2 - 2\mu v_0
% \end{align}
%
% This has the mean and variance
% \begin{align}
% 	\mu_y = \frac{1}{2}\left(1 - \frac{\varsigma\sqrt{2}(\mu - v_0)}{\sqrt{\pi}(\sigma^2 - \varsigma^2)}\right)
% \end{align}
\subsection*{Code}
\begin{verbatim} 
% params of sigmoid
%
v0 = 6;
r = 0.56;
varsigma = 1.699./r;

% params of Gaussian to be transformed - can play with these guys to test
% results
%
mu = 5.75;      
sigma = 2.5;

% for a numerical test
%
NSamples = 100000;
x = mu + sigma*randn(1,NSamples);
g_x = 0.5*(1 + erf((x-v0)/(sqrt(2)*varsigma)));

% transformed distribution
%
N_points = 10000;
y = linspace(1e-16,1-1e-16,N_points);
z = erfinv(2*y-1);
exponent_numerator = 2*sigma^2*z.^2 - (sqrt(2)*varsigma*z + v0 - mu).^2;
f_Y_y = exp(exponent_numerator/(2*sigma^2))*varsigma/sigma;

% 'a', 'b', and 'c' are from an expansion of 'exponent_numerator'
%
a = 2*(sigma^2 - varsigma^2);
b = -2*sqrt(2)*varsigma*(v0 - mu);
c = -mu^2 - v0^2 + 2*mu*v0;

% these are defined to complete the square
% 
k = c - b^2/(4*a);
h = - b/(2*a);

% alternate form of dist
%
f_Y_y2 = (varsigma/sigma)*exp(k/(2*sigma^2))*exp((z - h).^2 / (2*sigma^2 / a));

% this is supposed to be the mode of f_Y_y (its wrong though)
%
dist_mode = 0.5*(erf(sqrt(2)*varsigma*(mu-v0)/(2*(varsigma^2 - sigma^2))) + 1);
dist_mean = 0.5*(erf((mu-v0)/(sqrt(2*(varsigma^2 - sigma^2)))) + 1);

% plot results
%
figure
BinSize = 0.01;
bins = 0:BinSize:1;
N = hist(g_x,bins);
bar(bins,N/(sum(N)*BinSize),'facecolor','w','edgecolor','k')
hold on
xlim([0 1])
plot(y,f_Y_y2,...
    [dist_mode dist_mode],[0 max(f_Y_y)],...
    [dist_mean dist_mean],[0 max(f_Y_y)],...
    [approx_mean approx_mean],[0 max(f_Y_y)])
legend('numerical f_Y(y)','analytic f_Y(y)','analytic mode','analytic mean')
\end{verbatim}

\small
\bibliographystyle{ieeetr}
\bibliography{References}




\end{document}
